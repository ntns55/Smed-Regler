\chapter{Niveau 1}
Man kan komme langt ved at lave hestesko og søm, det er dog ikke noget spændende arbejde, men med flid og dedikation kan du også lappe rustning. Det tager dog sin tid.
\begin{table}[H]
    \centering
    \begin{tabular}{|p{0.50\textwidth}|p{0.25\textwidth}|}
    \rowcolor{cerulean!80}\hline
        Evne navn & Pris i XP \\\hline
         Brug af Tårnskjold & 2 \\\hline
         Ekstra NK Niv. 1 & 1 \\\hline
         Indkomst & 1\\\hline
         Lave låse Niv. 1 & 2\\\hline
         Reparere rustning Niv. 1 & 1\\\hline
    \end{tabular}
\end{table}

\section{Evne beskrivelse}
\subsection{Brug af Tårnskjold}
Du kan nu bruge tårnskjolde.

\subsection*{Ekstra NK Niv. 1}
\addcontentsline{toc}{subsection}{Ekstra NK Niv. 1}
Du har et ekstra nævekamp.\\

\subsection*{Indkomst}
\addcontentsline{toc}{subsection}{Indkomst}
Rul en 6-sidet terning ved tjek-in og få den mængde i Fjend\footnote{Møntfoden i A'kastin.} udleveret.

\subsection{Lave låse Niv. 1}
Smeden kan nu lave låse i Niv. 1. Disse låse kan sættes på både punge, tasker og kister, for at beskytte dem fra uvedkommende. Dette markeres med et gult bånd.
Nedenfor ses en beskrivelse af hvad det kræver at fremstille en lås.\\

\begin{table}[H]
    \centering
    \begin{tabular}{|p{0.25\textwidth}|p{0.25\textwidth}|}
    \hline
    \rowcolor{cerulean!80}
    \multicolumn{2}{c}{Lås Niv. 1}\\
    \hline
    \rowcolor{cerulean!40}
         Ressourcer & Tid \\\hline
         2 Jern & 10 min\\\hline
    \end{tabular}
    \end{table}

\subsection*{Reparere rustning Niv. 1}
\addcontentsline{toc}{subsection}{Reparere rustning Niv. 1}
Du kan genoprette 1 RP på ødelagt rustning, når du benytter 2 min på at reparere denne. Dette kan gentages indtil rustningen fremstår 'uden skader'.\\